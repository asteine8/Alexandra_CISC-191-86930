\chapter*{Preface to the Third Edition}

We have designed this third edition of {\it Java, Java, Java} to be
suitable for a typical Introduction to Computer Science (CS1) course
or for a slightly more advanced Java as a Second Language course. This
edition retains the ``objects first'' approach to programming and
problem solving that was characteristic of the first two editions.
Throughout the text we emphasize careful coverage of Java language
features, introductory programming concepts, and object-oriented
design principles.

The third edition retains many of the features of the first two
editions, including:

\begin{itemize}
\item Early Introduction of Objects
\item Emphasis on Object Oriented Design (OOD)
\item Unified Modeling Language (UML) Diagrams
\item Self-study Exercises with Answers
\item Programming, Debugging, and Design Tips.
\item {\em From the Java Library} Sections
\item {\em Object-Oriented Design} Sections
\item End-of-Chapter Exercises
\item Companion Web Site, with Power Points and other Resources
\end{itemize}

\noindent The {\em In the Laboratory} sections from the first two
editions have been moved onto the book's Companion Web Site. Table~1
shows the Table of Contents for the third edition.

\begin{table}[h!]
\caption{Table of Contents for the Third Edition.}
\begin{tabular}{|l|l|} \hline
{\bf Chapter} & {\bf Topic}       \cr \hline
Chapter 0 &  Computers, Objects, and Java ({\color{cyan}revised})\\
Chapter 1 &  Java Program Design and Development  \\
Chapter 2 &  Objects: Defining, Creating, and Using \\
Chapter 3 &  Methods: Communicating with Objects ({\color{cyan}revised}) \\
Chapter 4 &  Input/Output: Designing the User Interface ({\color{cyan}new}) \\
Chapter 5 &  Java Data and Operators \\
Chapter 6 &  Control Structures \\
Chapter 7 &  Strings and String Processing \\
Chapter 8 &  Inheritance and Polymorphism  ({\color{cyan}new}) \\
Chapter 9 &  Arrays and Array Processing \\
Chapter 10 &  Exceptions: When Things Go Wrong \\
Chapter 11 &  Files and Streams \\
Chapter 12 &  Recursive Problem Solving \\
Chapter 13 &  Graphical User Interfaces  \\
Chapter 14 & Threads and Concurrent Programming \\
Chapter 15 & Sockets and Networking ({\color{cyan}expanded}) \\
Chapter 16 &  Data Structures: Lists, Stacks, and \\ 
           &  \,\,\,\,Queues ({\color{cyan}revised and expanded}) \\
\hline 
\end{tabular}
\end{table}

\section*{What's New in the Third Edition}

The third edition has the following substantive changes:

\begin{itemize}

\item Although the book retains its emphasis on a ``running example''
that is revisited in several chapters, the CyberPet examples have been
replaced with a collection of games and puzzle examples. The CyberPet
examples from earlier editions will be available on the Companion Web
Site.

\item Chapters 0 (Computers, Objects, and Java) and 1 (Java Program
Design and Development) have been substantially reorganized and
rewritten. The new presentation is designed to reduce the pace with
which new concepts are introduced. The treatment of object-oriented
(OO) and UML concepts has also been simplified, and some of the more
challenging OO topics, such as polymorphism, have been moved to a
new Chapter 8.

\item The new Java 1.5 {\tt Scanner} class is introduced in Chapter 2
and is used to perform simple input operations. 

\item Chapter 4 (Input/Output: Designing the User Interface) has been
completely written. Rather than relying primarily on applet
interfaces, as in the second edition, this new chapter provides
independent introductions to both a command-line interface and a
graphical user interface (GUI). Instructors can choose the type of
interface that best suits their teaching style. The command-line
interface is based on the {\tt BufferedReader} class and is used
throughout the rest of the text. The GUI is designed to work with
either graphical applications or applets.  Both approaches are
carefully presented to highlight the fundamentals of user-interface
design.  The chapter concludes with an optional section that
introduces file I/O using the new {\tt Scanner} class.

\item Much of the discussion of inheritance and polymorphism, which
was previously woven through the first five chapters in the second
edition, has been integrated into a new Chapter 8.

\item An optional {\em graphics track} is woven throughout the text.
Beginning with simple examples in Chapters 1 and 2, this track also
includes some of the examples that were previously presented in
Chapter 10 of the second edition.

\item Chapter 15, on Sockets and Networking, is expanded to cover some
of the more advanced Java technologies that have emerged, including
servlets and Java Server Pages.

\item Chapter 16, on Data Structures, has been refocused on how to use
data structures. It makes greater use of Java's Collection Framework,
including the {\tt LinkedList} and {\tt Stack} classes and the {\tt
List} interface. It has been expanded to cover some advanced data
structures, such as sets, maps, and binary search trees.

\end{itemize}

\section*{The Essentials Edition}

An {\em Essentials Edition} of the third edition, which will include
Chapters 0-12, will be published as a separate title. The Essentials
Edition will cover those topics (Chapters 0-9) that are covered in
almost all introductory (CS1) courses, but it will also include topics
(Exceptions, File I/O, and Recursion) that many CS1 instructors have
requested.

\section*{Why Start with Objects?}

The Third Edition still takes an {\em objects-early} approach to
teaching Java, with the assumption that teaching beginners the ``big
picture'' early gives them more time to master the principles of
object-oriented programming.  This approach seems now to have gained
in popularity as more and more instructors have begun to appreciate
the advantages of the object-oriented perspective.

Object Orientation (OO) is a fundamental problem solving and design
concept, not just another language detail that should be relegated to
the middle or the end of the book (or course).  If OO concepts are
introduced late, it is much too easy to skip over them when push comes
to shove in the course.

The first time I taught Java in our CS1 course I followed the same
approach I had been taking in teaching C and C++ --- namely, start
with the basic language features and structured programming concepts
and then, somewhere around midterm, introduce object orientation.
This approach was familiar, for it was one taken in most of
the textbooks then available in both Java and C++.

One problem with this approach was that many students failed to get
the big picture.   They could understand loops, if-else constructs,
and arithmetic expressions, but they had difficulty decomposing a
programming problem into a well-organized Java program.  Also, it
seemed that this procedural approach failed to take advantage of the
strengths of Java's object orientation.  Why teach an object-oriented
language if you're going to treat it like C or Pascal?

I was reminded of a similar situation that existed when Pascal was the
predominant CS1 language.  Back then the main hurdle for beginners was
{\it procedural abstraction} --- learning the basic mechanisms
of procedure call and parameter passing and learning how to {\bf
design} programs as a collection of procedures.  {\it Oh! Pascal!},
my favorite introductory text, was typical of a ``procedures early''
approach.  It covered procedures and parameters in Chapter 2, right
after covering the assignment and I/O constructs in Chapter 1. It then
covered program design and organization in Chapter 3. It didn't get
into loops, if-else, and other structured programming concepts until
Chapter 4 and beyond.

Today, the main hurdle for beginners is the concept of {\it object
abstraction}. Beginning programmers must be able to see a program as a
collection of interacting objects and must learn how to decompose
programming problems into well-designed objects.  Object orientation
subsumes both procedural abstraction and structured programming
concepts from the Pascal days.  Teaching objects-early takes a
top-down approach to these three important concepts.  The sooner you
begin to introduce objects and classes, the better the chances that
students will master the important principles of object orientation.

Java is a good language for introducing object orientation.  Its
object model is better organized than C++. In C++ it is easy to ``work
around'' or completely ignore OO features and treat the language like
C. In Java there are good opportunities for motivating the discussion
of object orientation.  For example, it's almost impossible to discuss
GUI-based Java applications without discussing inheritance and
polymorphism.  Thus rather than using contrived examples of OO
concepts, instructors can use some of Java's basic features ---
the class library, Swing and GUI components --- to motivate these
discussions in a natural way.

\section*{Organization of the Text}

The book is still organized into three main parts. Part I (Chapters
0-4) introduces the basic concepts of object orientation and the basic
features of the Java language.  Part II (Chapters 5-9) focuses on
remaining language elements, including data types, control structures,
string and array processing, and inheritance and polymorphism.  Part
III (Chapters 10-16) covers advanced topics, including exceptions,
file I/O, recursion, GUIs, threads and concurrent programming, sockets
and networking, data structures, servlets, and Java Server Pages.

The first two parts make up the topics that are typically covered in
an introductory CS1 course. The chapters in Part III are
self-contained and can be selectively added to the end of a CS1 course
if time permits.  

The first part (Chapters 0 through 4) introduces the basic concepts of
object orientation, including objects, classes, methods, parameter
passing, information hiding, and a little taste of inheritance, and
polymorphism.  The primary focus in these chapters is on introducing
the basic idea that an object-oriented program is a collection of
objects that communicate and cooperate with each other to solve
problems. Java language elements are introduced as needed
to reinforce this idea.  Students are given the basic building blocks
for constructing Java programs from scratch.  

Although the programs in the first few chapters have limited
functionality in terms of control structures and data types, the
priority is placed on how objects are constructed and how they
interact with each other through method calls and parameter passing.

The second part (Chapters 5 through 9) focuses on the remaining
language elements, including data types and operators (Chapter 5),
control structures (Chapter 6), strings (Chapter 7), and arrays
(Chapter 9).  It also provides thorough coverage of inheritance and
polymorphism, the primary mechanisms of object orientation: (Chapter
8).  

Part three (Chapters 10 through 16) covers a variety of advanced
topics (Table~1).  Topics from these chapters can be used selectively
depending on instructor and student interest.

Throughout the book, key concepts are introduced through simple,
easy-to-grasp examples.  Many of the concepts are used to create a set
of games, which are used as a running example throughout the text.
Our pedagogical approach focuses on design. Rather than starting of
with language details, programming examples are carefully developed
with an emphasis on the principles of object-oriented design.

Table\ref{tab-course} provides an example syllabus from our
one-semester CS1 course.  Our semester is 13 weeks (plus one reading
week during which classes do not meet). We  pick and choose from among
the advanced topics during the last two weeks of the course, depending
on the interests and skill levels of the students.

\vspace*{2pc}
\noindent Ralph Morelli\\
\date{\today}


\begin{table}\caption{A one-semester course.
\label{tab-course}}
\offinterlineskip
\def\mstrut{\vrule width 0pt height 10pt depth 3pt}%
\halign{\mstrut
\quad #\hfil\quad \vrule&
\quad #\hfil\quad \vrule&
\quad #\hfil\quad \cr
%\multispan{3}\hfil{A one-semester course.}\hfil\cr
\multispan{3}\hfil{}\hfil\cr
{\bf Weeks} & {\bf Topics}                  & {\bf Chapters} \cr
\noalign{\hrule}
   1           & Object Orientation, UML         &  Chapter 0    \cr
               & Program Design and Development  &  Chapter 1     \cr
   2-3         & Objects and Class Definitions &  Chapter 2      \cr
               & Methods and Parameters        &  Chapter 3      \cr
               & Selection structure (if-else) &                 \cr
   4           & User Interfaces and I/O       &  Chapter 4      \cr
   5           & Data Types and Operators      &  Chapter 5      \cr
   6--7        & Control Structures (Loops)    &  Chapter 6      \cr
               & Structured Programming        &                 \cr
   8           & String Processing (loops)     &  Chapter 7      \cr
   9           & Inheritance and Polymorphism    &  Chapter 8      \cr
   10          & Array Processing              &  Chapter 9      \cr
   11          & Recursion                     &  Chapter 12     \cr
   12          & Advanced Topic (Exceptions)   &  Chapter 10      \cr
   13          & Advanced Topic (GUIs)         &  Chapter 11     \cr
               & Advanced Topic (Threads)      &  Chapter 15     \cr
}\end{table}

