\setcounter{table}{0}
\setcounter{figure}{0}
\renewcommand{\thetable}{\mbox{C.\arabic{table}}}%
\renewcommand{\thefigure}{\mbox{C--\arabic{figure}}}%

\chapter{The ASCII and Unicode Character Sets}

\markboth{{\color{cyan}APPENDIX\,C\,\,$\bullet$\,\,}ASCII and Unicode Character Sets}
{{\color{cyan}APPENDIX\,C\,\,$\bullet$\,\,}ASCII and Unicode Character Sets}

\noindent Java uses version 2.0 of the Unicode character set for
representing character data.  The Unicode set represents each character
as a 16-bit unsigned integer.  It can, therefore, represent 2$^{16}$ $=$
65,536 different characters.  This enables Unicode to represent
characters from not only English but also a wide range of international
languages.  For details about Unicode, see

\begin{jjjlisting}
\begin{lstlisting}[commentstyle=\color{black}]
http://www.unicode.org
\end{lstlisting}
\end{jjjlisting}

\vspace{-3pt}Unicode supersedes the ASCII character set (American Standard Code for
Information Interchange).  The ASCII code represents each character as
a 7-bit or 8-bit unsigned integer.   A 7-bit code can represent only
2$^7$ $=$ 128 characters.   In order to make Unicode backward compatible
with ASCII, the first 128 characters of Unicode have the same
integer representation as the ASCII characters.

Table~C.1 shows the integer representations for the
{\it printable} subset of ASCII characters.  The characters with codes
0 through 31 and code 127 are {\it nonprintable} characters, many of
which are associated with keys on a standard keyboard.  For example,
the delete key is represented by 127, the backspace by 8, and the
return key by 13.

%%%Char   SP !  "  #  $  %  &  \mbox{\fontTimes\char39}\,\,  (  )   *  +  ,  -  .  /

\begin{table}[h!]
%\hphantom{\caption{ASCII codes for selected characters}}
\TBT{0pc}{ASCII\index{ASCII} codes for selected characters}
%%%\vspace{-6pt}
\begin{jjjlisting}
\begin{lstlisting}[stringstyle=\color{black}]
 Code   32 33 34 35 36 37 38 39 40 41 42 43 44 45 46 47
 Char   SP !  "  #  $  %  &  '  (  )   *  +  ,  -  .  /

 Code   48 49 50 51 52 53 54 55 56 57
 Char   0  1  2  3  4  5  6  7  8  9

 Code   58 59 60 61 62 63 64
 Char   :  ;  <  =  >  ?  @

 Code   65 66 67 68 69 70 71 72 73 74 75 76 77
 Char   A  B  C  D  E  F  G  H  I  J  K  L  M

 Code   78 79 80 81 82 83 84 85 86 87 88 89 90
 Char   N  O  P  Q  R  S  T  U  V  W  X  Y  Z

 Code   91 92 93 94 95 96
 Char   [  \  ]  ^  _  `

 Code   97 98 99 100 101 102 103 104 105 106 107 108 109
 Char   a  b  c  d   e   f   g   h   i   j   k   l   m

 Code   110 111 112 113 114 115 116 117 118 119 120 121 122
 Char   n   o   p   q   r   s   t   u   v   w   x   y   z

 Code   123 124 125 126
 Char   {   |   }   ~
\end{lstlisting}
\end{jjjlisting}
\end{table}

