\setcounter{table}{0}
\setcounter{figure}{0}
\renewcommand{\thetable}{\mbox{G.\arabic{table}}}%
\renewcommand{\thefigure}{\mbox{G--\arabic{figure}}}%

\chapter{Java Autoboxing and Enumeration}

\markboth{{\color{cyan}APPENDIX\,G\,\,$\bullet$\,\,}Java Autoboxing and Enumeration}
{{\color{cyan}APPENDIX\,G\,\,$\bullet$\,\,}Java Autoboxing and Enumeration}

\noindent This appendix describes some basic properties of {\em autoboxing}
and {\em enumeration}, two of the features added to the Java language
with the release of Java 5.0.  As for many language features, there
are details and subtleties involved in using these features that are
not covered here.  For further details, you should consult Sun's
online references or other references for a more comprehensive
description.

\section*{Autoboxing and Unboxing}
\noindent {\bf Autoboxing} refers to the automatic storing of a value
of primitive type into an object of the corresponding wrapper class.
Before autoboxing, it was necessary to explicitly box values into wrapper 
class objects with code like:

\begin{jjjlisting}
\begin{lstlisting}
Integer iObj = new Integer(345);
double num = -2.345;
Double  dObj = new Double(num);
\end{lstlisting}
\end{jjjlisting}

\noindent Java 5.0 automatically creates a wrapper class object from a
value of primitive type in many situations where a wrapper class object
is expected.  The assignments above can be accomplished with the simpler code:

\begin{jjjlisting}
\begin{lstlisting}
Integer iObj = 345;
double num = -2.345;
Double  dObj = num;
\end{lstlisting}
\end{jjjlisting}

There is a corresponding feature in Java 5.0 which automatically performs 
the {\bf unboxing} of primitive values from wrapper class objects.
Instead of the explicit unboxing in:

\begin{jjjlisting}
\begin{lstlisting}
int m = iObj.intValue();
double x = dObj.doubleValue();
\end{lstlisting}
\end{jjjlisting}

\noindent Java 5.0 allows the simpler:

\begin{jjjlisting}
\begin{lstlisting}
int m = iObj;
double x = dObj;
\end{lstlisting}
\end{jjjlisting}

Java 5.0 provides autoboxing of primitive values and automatic unboxing of wrapper 
class objects in expressions or in arguments of methods, where such a conversion
is needed to complete a computation.  Beginning programmers are unlikely to
encounter many problems that require such conversions.  One situation which often
requires boxing and unboxing are applications that involve data structures.  The 
generic type data structures of Chapter 16 must store objects but the data to
be stored might be represented as values of a primitive type.  The code segment
below should give you some idea of the type of situation where autoboxing and 
unboxing can be a genuine help simplifying one's code: 
 
\begin{jjjlisting}
\begin{lstlisting}
Stack<Integer> stack = new Stack<Integer>();
for(int k = -1; k > -5; k--)
    stack.push(k);
while (!stack.empty())
    System.out.println(Math.abs(stack.pop()));
\end{lstlisting}
\end{jjjlisting}

\noindent  Notice that the {\tt stack.push(k)} method is expecting an
{\tt Integer} object so the {\tt int} value stored in {\tt k} will be
autoboxed into such an object before the method is executed.  Also note that
the {\tt Math.abs()} method in the last line of the code fragment is
expecting a value of primitive type so the {\tt  Integer} value returned
by {\tt stack.pop()} must be automatically unboxed before the {\tt Math.abs()} 
method can be applied.

Sun's online Java 5.0 documentation can be consulted for a more precise description 
of where autoboxing and unboxing takes place and a list of some special situations 
where code allowing autoboxing can lead to confusion and problems.

\section*{Enumeration}
\noindent A new {\bf enumeration} construct was included in Java 5.0 to make 
it simpler to represent a finite list of named values as a type. The
{\bf enum} keyword was added as part of this construct. Standard
examples of lists of values appropriate for enumerations are the days
of the week, months of the year, the four seasons, the planets, the
four suits in a deck of cards, and the ranks of cards in a deck of
cards. The following declaration of {\tt Season} enumerated type is an
example used by the Sun online documentation.

\begin{jjjlisting}
\begin{lstlisting}
public enum Season {spring, summer, fall, winter}
\end{lstlisting}
\end{jjjlisting}

\noindent Compiling a file that contains only this statement will create a
{\tt Season.class} file that defines a Java type just in the same way
that compiling class definitions does.  The variables and values of type
{\tt Season} can be referred to in other classes just like other types and
values. For example, the following statements are valid statements in 
a method definition in another class:

\begin{jjjlisting}
\begin{lstlisting}
Season s1 = winter;
if (s1 == spring)
    System.out.println(s1);    
\end{lstlisting}
\end{jjjlisting}

\noindent Note that the values of enumerated types can be used in 
assignment  statements, equality relations, and it will be printed out exactly
as declared in the {\tt enum} statement.

The {\tt enum} declaration could also occur inside the definition of a
class and be declared as either {\tt public} or {\tt private}. In this case the 
visibility of the type would be determined in a manner similar to inner classes.

A standard way to represent such a finite list of values in Java before the
{\tt enum} construct was to create a list of constants of type {\tt int}.
For example, if one wanted to represent the four seasons you would have to do 
it inside a definition of a class, say of a class named {\tt Calendar}.
Such a representation might look like:

\begin{jjjlisting}
\begin{lstlisting}
public class Calendar {
    public static final int SPRING = 0;
    public static final int SUMMER = 1;
    public static final int FALL = 2;
    public static final int WINTER = 3;
    
    // Other Calendar definitions
    
} // Calendar
\end{lstlisting}
\end{jjjlisting}

\noindent In addition to being a lengthier declaration, note that other classes 
that wish to refer to this representation would have to use notation something like:
 
\begin{jjjlisting}
\begin{lstlisting}
int s1 = Calendar.WINTER;
if (s1 == Calendar.SPRING)
    System.out.println(s1);    
\end{lstlisting}
\end{jjjlisting}
 
\noindent In addition to being more awkward, note that the {\tt println()}
call will print out an integer in this case.  Some additional code would have 
to be written to be able to print the names of the seasons from the
{\tt int} values used to represent them.  It is the case that for methods in the
{\tt Calendar} class, the names of the constants look very much like the
values of the {\tt enum} type.

To illustrate a couple of additional advantages of the {\tt enum}
structure, lets consider using the {\tt int} representation above in a method
definition that describes the start date of a given season. Code for such a method
would look something like:

\begin{jjjlisting}
\begin{lstlisting}
    public static String startDate(int s){
        switch(s){
        case SPRING : return "Vernal Equinox";
        case SUMMER : return "Summer Solstice";
        case FALL : return "Autumnal Equinox";
        case WINTER : return "Winter Solstice";
        } // switch
        return "error";
    } // startDate()   
\end{lstlisting}
\end{jjjlisting}

\noindent This method has a problem referred to as not being {\bf typesafe}.
We would want the {\tt startDate()} method to be called only with an
argument equal to an {\tt int} value of 0, 1, 2,  or 3.  There is no
way to tell the compiler to make sure that other {\tt int} values
are not used as an argument to this method.

Let's contrast this with a similar {\tt startDate()} method that
can refer to the {\tt Season} enumerated type that was defined above. 
The {\tt Calendar} class (Figure~G--1) shows the definition of 
a {\tt startDate()} method as well as a method to print a list of 
seasons with corresponding starting dates.
\begin{figure}[h!]
\jjjprogstart
\begin{jjjlisting}
\begin{lstlisting}
public class Calendar {
    public static String startDate(Season s){
        switch(s){
        case spring : return "Vernal Equinox";
        case summer : return "Summer Solstice";
        case fall : return "Autumnal Equinox";
        case winter : return "Winter Solstice";
        } // switch
        return "error";
    } // startDate()
 
    public static void printDates(){
        for (Season s : Season.values())
            System.out.println(s + " - " + startDate(s));
    } // printDates()
} // Calendar
\end{lstlisting}
\end{jjjlisting}
\jjjprogstop{A Calendar class using the {\tt Season}.}
{fig-calendar}
\end{figure}
Note that the parameter of {\tt startDate()} is of type {\tt Season}
and so the Java compiler can check that call to this method have
an argument of this type.  This time the {\tt startDate()} is typesafe.

The {\tt printDates()} method illustrates another feature of the 
enumeration structure.  The {\tt for} loop in this method is the 
{\tt for-in} loop which was added to Java 5.0.  The expression
{\tt Season.values()} denotes a list of the elements of the
type in the order that they were declared.  The {\tt for-in} loop
iterates through all the values of the type in the correct order and,
in this case, prints out the type name followed by a dash followed
by the {\tt String} computed by the {\tt startDate()} method. 
The output when the {\tt printDates()} method is called is given below: 

\begin{jjjlisting}
\begin{lstlisting}
spring - Vernal Equinox
summer - Summer Solstice
fall - Autumnal Equinox
winter - Winter Solstice
\end{lstlisting}
\end{jjjlisting}

The {\tt for-in} loop provides a very nice way to iterate through the
values of any enumerated type. You may wish to write a corresponding 
method for the  earlier {\tt int} representation of the seasons for a
comparison.

Sun's online Java 5.0 documentation provides a more precise definition of 
enumerated types and describes quite a number of other features that we
have not alluded to.


